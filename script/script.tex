\documentclass{article}

\usepackage{amsmath}
\usepackage{blindtext}
\usepackage{hyperref}

\hypersetup{
    colorlinks=true,
    urlcolor=cyan,
    }

\title{Seminário de álgebra I}
\author{Paulo Artur, \ldots?}

\begin{document}

\maketitle

\section*{Introdução}

Apresentação dos palestrantes

Hoje vamos fazer uma introdução de como shaders funcionam e sua relação com álgebra linear.

\begin{itemize}
\item Primeiro vamos falar como os computadores representam cores
\item \ldots?
\item Definir o que são shaders, e explicar sua relação com matrizes
\end{itemize}

\section*{Representação de cores}

TALVEZ vamos explicar como o sistema de cores evoluiu durante os anos.

\subsection*{Sistema Grayscale}

Aqui vamos explicar como o sistema grayscale funciona.

O sistema grayscale é um sistema de cor que por um único valor numérico podemos representar quão branco ou quão preto a luz é.

Normalmente em computação gráfica existe 2 maneiras de representar os valores numéricos referentes a cor.

\begin{itemize}
\item de 0 até 255, ou representação 8-bits
\item de 0.0 até 1.0, ou representação decimal normalizada, que pode ser conseguida por essa fórmula $\frac{\text{ValorDaCor}}{\text{ValorTotal}}$
\end{itemize}


Nesse documento vamos usar a representação decimal normalizada.

\subsection*{Sistema RGB}

Aqui vamos explicar como o sistema RGB funciona.

O sistema rgb é como o sistema grayscale, com a diferença que no sistema grayscale você tem apenas 1 valor numérico (canal) para representar os valores que vão de 
preto até branco. No sistema rgb vocẽ tem 3 valores numérios (canais) que misturados representam qualquer cor.

\begin{itemize}
    \item r: preto até vermelho 
    \item g: preto até verde 
    \item b: preto até azul 
\end{itemize}

(Mostrar animação 1).

\section*{Shaders}

Aqui explicamos o que são shaders e pixels

Em monitores modernos existe uma matriz ou grid de pequenos pontos que são iluminados e coloridos, cada um desses pontos é chamado de pixel. Cada um desses pixeis tem um valor
RGB atribuído.

Cada ponto em uma imagem ou textura é chamado de texel mas informalmente podemos nos referir a esses pontos como pixeis.

Um shader qualquer programa que roda na GPU, há vários tipos de shaders.

Vamos falar sobre o fragment/pixel shader, que é um shader instanciado para cada pixel na tela.

Temos uma matriz de números, existe uma lei de formação para a geração dessa matriz. Agora imagine que essa matriz de números seja uma matriz de pixels -- Uma imagem -- e que
a lei de formação dessa matriz seja o shader. É assim que o fragment shader funciona.

(Mostrar animação 2).

\subsection*{Exemplos de shaders}

Mostrar esses exemplos e dar um overview de como eles funcionam

\begin{itemize}
\item O exemplo mais simples de shader:\\
 \url{https://www.shadertoy.com/view/4fKBWw}

\item Circulo:\\
 \url{https://www.shadertoy.com/view/XcVfWw}

\item Shader animado:\\
 \url{https://www.shadertoy.com/view/XfKBDw}

\item Shader mais complexo:\\
 \url{https://www.shadertoy.com/view/Xclcz2}

\end{itemize}

\end{document}

